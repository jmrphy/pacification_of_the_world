\begin{abstract} 
This manuscript collects a variety of theoretical and empirical notes on the relationship between information technology and political contention. On the side of information technology, I am most interested in how information technology triggers cultural (ethical and communicative) shifts with political implications; on the side of political contention, I am most interested in the rise and fall of institutional or system-level political contention in the modern period and the postwar period especially. In the most general terms, I argue that advancements in information technology tend to empower challengers of a status quo in the short-run but, through multiple causal pathways, ultimately strengthen the political stability of a status quo in the long-run. My arguments potentially explain a wide variety of diverse puzzles, such as why there was massive system-level political contention in the 1960s and how it was so rapidly pacified at the end of the 1960s, why economic inequality persists under formally democratic institutions, and why we observe increasing prevalence of social anomie, political mistrust, and mental illness in the wealthy liberal democracies.
\end{abstract}