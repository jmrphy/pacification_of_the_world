\begin{abstract} 
Though a great deal has been written on diverse political implications of information technology, and models of political economy now routinely include information considerations, there does not yet exist a general political theory of information technology. This manuscript outlines a series of theoretical and empirical notes to this end. In the most general terms, I argue that advancements in information technology tend to empower challengers of a status quo in the short-run but, through multiple causal pathways, ultimately strengthen the political stability of a status quo in the long-run. My argument explains why nearly all dominant political categories have come to ring hollow, diverse types of social trust have been depleted, and system-level political contention has been pacified and erased from memory in the wealthy liberal world since the 1970s.


One reason is that improvements in information are a type of "skill-biased technological change" in which the relatively more educated see their power increase more than the less educated, but this is not unique to information technology. The crucial and unique implication is that, at least within market societies, advancements in information-processing bias all social communication toward status quo assumptions. Despite the widespread belief that new information technologies enhance democracy and cooperation, I argue that in the long-run, improvements in information-processing increase the profitability of instrumental or exploitative communication relative to cooperative or goal-driven communication. This is because those most invested in the status quo will use advancements in information-processing to advance themselves within the status quo rather than change the status quo as such; this forces those who would otherwise seek to change the status quo to speak and behave in agreement with the assumptions of the status quo in order to survive. While such a dynamic emerges for many types of technological innovation, only improvements in information-processing have such unique and far-reaching negative externalities on the public sphere because, while skill-biased technological change in industry might lead some workers to earn more in the factor, skill-biased change in information-processing leads some individuals to earn more from everyone in their life in general. For these reasons, the information revolution sparked a process in which all communication around the world is increasingly only instrumental communication, and the substantial communication required for individuals and communities to challenge elites is increasingly evacuated. This accounts for why, today, nearly all dominant political categories ring hollow, diverse types of social trust are depleted, and militant political resistance to the status quo has been pacified and erased from memory in the short span of about thirty years.
\end{abstract}