\begin{abstract} 
Though a great deal has been written on diverse political implications of information technology, and models of political economy now routinely include information considerations, there does not yet exist a general political theory of information technology. This manuscript outlines a series of theoretical and empirical notes to this end. In the most general terms, I argue that advancements in information technology tend to empower challengers of a status quo in the short-run but, through multiple causal pathways, ultimately strengthen the political stability of a status quo in the long-run. My argument explains why nearly all dominant political categories have come to ring hollow, diverse types of social trust have been depleted, and system-level political contention has been pacified and erased from memory in the wealthy liberal world since the 1970s.
\end{abstract}